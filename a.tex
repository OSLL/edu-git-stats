\documentclass{beamer}

\usepackage{polyglossia}
\usepackage{fontspec}
\usepackage{nameref}
\usepackage{ifthen}
\usepackage{minted}
\usepackage[math-style=ISO,vargreek-shape=unicode]{unicode-math}

\setdefaultlanguage[spelling=modern,babelshorthands=true]{russian}
\setotherlanguage{english}

\defaultfontfeatures{Ligatures={TeX}}
\setmainfont{CMU Serif}
\setsansfont{CMU Sans Serif}
\setmonofont{CMU Typewriter Text}  
\setmathfont{Latin Modern Math}
\AtBeginDocument{\renewcommand{\setminus}{\mathbin{\backslash}}}

\makeatletter
\newcommand*{\currentname}{\@currentlabelname}
\makeatother
\def\t{\texttt}

\newcommand{\cimg}[2]{%
	\begin{center}%
		\ifthenelse{\equal{#2}{}}{%
			\includegraphics[width=0.75\linewidth]{#1}
		}{%
			\includegraphics[width=#2\linewidth]{#1}
		}%
	\end{center}%
}

\newmintinline[cinl]{c}{} %\c is defined :(
\newmintinline[cpp]{cpp}{}
\newmintinline[python]{python}{}
\newmintinline[bash]{bash}{}
\newmintinline[make]{make}{}
\setminted{obeytabs,tabsize=4,linenos,autogobble}
\newminted{c}{}
\newminted{cpp}{}
\newminted{python}{}
\newminted{bash}{}
\newminted{make}{}

\usetheme{Antibes}
\useoutertheme{infolines_foot}
\setbeamercovered{transparent=20}

\title[gitstat]{gitstat \\ Анализ и статистика по истории \t{git}}
\author[Лапшин Дмитрий]{Лапшин Дмитрий \\ Руководитель: Кринкин Кирилл Владимирович}
\institute{СПб НИАУ РАН}
\date{Весна 2015}

\begin{document}

\begin{frame}
	\titlepage
\end{frame}

\section{Содержание}

\begin{frame}[t]{\currentname}
	\tableofcontents
\end{frame}

\section{Цели и задачи проекта}

\subsection{Цель проекта}

\begin{frame}[t]{\currentname}
	Реализовать анализатор, находящий в проекте проблемные места, исходя из истории проекта, отслеживаемой системой контроля версий \t{git}.
	\cimg{git.jpg}{0.5}
\end{frame}

\subsection{Задачи проекта}

\begin{frame}[t,fragile]{\currentname}
	\begin{enumerate}[<+->]
	\item
		Изучить систему контроля версий \t{git}, как она составляет изменения (\bash'git diff').
	\item
		Сбор истории файлов и строк из \t{git} (\bash'git show', \bash'git log', \bash'git blame', ...).
	\item
		Метод поиска проблемных строк в проекте на выданном диапазоне истории.
	\item
		Реализация анализатора файла, папки.
	\item
		Визуализация.
	\end{enumerate}
\end{frame}

\subsection{Проблемные строки}

\begin{frame}[t,fragile]{\currentname}
	\begin{cppcode}
		// 1 revision:
		double const g = 9.8;
	\end{cppcode}
	\begin{cppcode}
		// this day later:
		double const g = 9.80665;
	\end{cppcode}
	\begin{cppcode}
		// the day after:
		double const g = 10;
	\end{cppcode}
	\begin{cppcode}
		// next week:
		double const g = 2;
	\end{cppcode}
	\begin{itemize}
	\item<2->
		\textit{Не зная}, что означает \cpp'double const g', \textit{видно}, что что-то не так.
	\item<3->
		Это не видно, если смотреть только на \textit{текущую} версию кода.
	\end{itemize}
\end{frame}

\begin{frame}[t]{\currentname}
	\begin{itemize}
	\item<1->
		Проблемной имеет смысл назвать строку, которая за малое количество раз много менялась.
	\item<2->
		Если строку целиком заменили, удалили или добавили "--- это не интересно.
	\item<3->
		Строки сдвигаются по нумерации, но это их не меняет.
	\end{itemize}
	\pause[4]
	В качестве текущей характеристики <<проблемности>> строки "--- сколько суммарно изменений внутри строки произошло.
\end{frame}

\section{Деятельность}

\subsection{\t{git}}

\begin{frame}[t]{\currentname}
	В начале догло исследловался сам \t{git}, чтобы понять, что можно использовать прямо из \t{git}.
	\pause
	\begin{itemize}
	\item
		В качестве основной команды вывода истории используется стандартная \bash'git log' с кучей полезных флагов.
		\pause
	\item
		Параллельно с ней использовался другой вызов \bash'git log --word-diff', который выделяет извенения именно внутри строк.
		\pause
	\item
		Отслеживать историю конкретной строки \t{git} не умеет.
		\pause
	\item
		Для обхода этого анализатор разбирает вывод \bash'git log' и проходтися с помощью структуры \t{rope}
		(использовалось декартого дерево, \t{treap}).
	\end{itemize}
\end{frame}

\subsection{Объединение и разделение строк}

\begin{frame}[t,fragile]{\currentname}
	\t{word-diff} иногда начинает объединять или разделять строки:
	\pause
	\begin{ccode}
		r = ioctl(hdev->control, VHOST_SET_LOG_BASE, // Enter
		          (uint64_t)(unsigned long)hdev->log);
	\end{ccode}
	\pause
	\begin{minted}[linenos=false]{c}
		r = hdev->vhost_ops->vhost_call(hdev, VHOST_SET_LOG_BASE,
			hdev->log); // Одна строка
	\end{minted}
	\pause
	\visible<4->{
		\cimg{pain-2.png}{1}
	}
	\pause
	\visible<5->{
		\cimg{pain-1.png}{1}
	}
	\pause
	И если с таким случаем разобраться получилось...
\end{frame}

\begin{frame}[t,fragile]{\currentname}
	...то с таким "--- нет:
	\pause
	\begin{ccode}
		while ((bit = sizeof(log) > sizeof(int) ?
		        ffsll(log) : ffs(log))) {
	\end{ccode}
	\pause
	\begin{ccode}
		while (log) {
		    int bit = ctzl(log);
	\end{ccode}
	\pause
	Тут \bash'--word-diff' начинает смешивать строки в изменениях.
	\pause

	Как именно работает режим \bash'--word-diff' нигде не описано.
	\pause

	Кроме этого, были обнаружены ещё несколько странных особенностей вывода \bash'git log --word-diff'.
\end{frame}

\section{Результаты}
\subsection{Что удалось}

\begin{frame}[t]{\currentname}
	\pause
	\begin{itemize}
	\item
		Разобраться, как \t{git} работает с историей на довольно глубоком уровне.
		\pause
	\item
		В целом много узнать о механизмах \t{git}-а (взаимотоношения коммитов, идеология патчей, виды и режимы файлов...).
		\pause
	\item
		Разобрать крайне сложно устроенный вывод.
		\pause
	\item
		Реализация подсчёта изменений в одном файле реализована и работает, пока не налетает на ошибку выше.
	\end{itemize}
\end{frame}

\subsection{Что ещё не удалось / Планы на будущее}

\begin{frame}[t]{\currentname}
	Большинство этих задач можно решить в ближайшем будущем:
	\pause
	\begin{itemize}
	\item
		Анализировать сразу папки.
		\pause
	\item
		Визуализатор.
	\end{itemize}
	\pause
	Эти задачи требуют уже больших усилий:
	\begin{itemize}
	\item
		Разобрать анализ \t{git}-ом фрагментов, описанных выше.
		\pause
	\item
		Возможно, стоит рассмотреть дополнительные классы изменений (переименовывание файлов, ...) и улучшить разбиение на слова.
	\end{itemize}
\end{frame}

\subsection{Выводы}

\begin{frame}[t]{\currentname}
	\pause
	\begin{itemize}
	\item
		\begin{quote}
			git --- stupid container tracker
		\end{quote}
		\t{git} действительно мало понимает в отслеживаемых файлах.
		Для лучшего анализа его необходимо настраивать на каждый формат файла, который он хранит.
		\pause
	\item
		Реализация \t{git} практически не документирована.
		\pause
	\item
		Тем не менее, \t{git} "--- очень мощная система с множеством опций и методов модификации.
		\pause
	\item
		Задача анализа истории по \t{git} действительно сложна, но решаема.
		\pause
	\item
		Python "--- великолепный язык для взаимодействия с командами и их выводами.
	\end{itemize}
\end{frame}

\subsection{Чему я научлся}

\begin{frame}[t]{\currentname}
	\pause
	\begin{itemize}
	\item
		Глубокая работа с \t{git}.
		\pause
	\item
		Чтение большого объёма непонятной документации, которую никто не читает.
		\pause
	\item
		Регулярные выражения
		\pause
	\item
		Python
	\end{itemize}
\end{frame}

\section{}

\begin{frame}
	\begin{center}
		Спасибо за внимание.

		\bigskip
		Репозиторий: \url{https://github.com/OSLL/edu-git-stats}

		\bigskip
		{\footnotesize
		В качестве примеров были взяты строки из \\
		\url{https://github.com/OSLL/qemu-xtensa/}

		}
	\end{center}
\end{frame}

\end{document}
